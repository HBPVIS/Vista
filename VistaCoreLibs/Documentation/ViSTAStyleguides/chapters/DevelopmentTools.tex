%% $Id$

%% Copyright (c)  1998-2010
%% by  RWTH-Aachen, Germany
%% Some rights reserved.

%% This work is licensed under the Creative Commons Attribution-Share
%% Alike 3.0 License. To view a copy of this license, visit
%% http://creativecommons.org/licenses/by-sa/3.0/ or send a letter to
%% Creative Commons, 171 Second Street, Suite 300, San Francisco,
%% California, 94105, USA.

\section{Development Tools}

\subsection{SourceForge.net}
The development of the ViSTA core libraries is coordinated through the \code{vistavrtoolkit} project hosted at SourceForge.net:

\code{http://www.sourceforge.net/projects/vistavrtoolkit}

\subsubsection{Mailing List}
General user support as well as developer discussion happens via a dedicated mailing list hosted at SourceForge.net, the \code{vistavrtoolkit-general} list.
For details on how to register with the list, visit the \code{ViSTA} project page at the SourceForge.net project URL mentioned above.

\subsubsection{Bug Tracker}
We also host a bug tracker at SourceForge.net. 
It should be used by users and developers alike to report bugs and request features (as wishlist items).
You can find the bug tracker via the project site mentioned above, by navigating to Develop $\rightarrow$ Tracker $\rightarrow$ Bugs.

\subsection{Subversion}
Currently we use the open source software Subversion (http://subversion.apache.org/) for revision control.
There are several open clients available.
We recommend using either the CollabNet command-line client which comes with most linux distributions or the TortoiseSVN client on MS Windows.
Please refer to the documentation of the respective tools.
If you still have trouble accessing the SourceForge repository, you can drop a mail to the public mailing list (\code{vistavrtoolkit-general}) and get someone to help you.

\subsubsection{Commit mailing list}

Commits to the \code{vistavrtoolkit} repository will be mailed automatically to the \\
\code{vistavrtoolkit-commits@lists.sourceforge.net} mailing list.

If you want inline diffs, subscribe to the \code{vistavrtoolkit-commits-verbose} list instead.
The diff output size is truncated to 100000 characters, so that even on the verbose list, the message sizes will be around 100kB maximum.

\subsubsection{Subversion FAQ}

\begin{itemize}
\item[\textbf{Q}] I am working on a release branch of the ViSTA Core libs and detected an error/want to make a change.
  How do I accomplish this?
\item[\textbf{A}] There are several situations possible.

  \begin{itemize}
  \item In case of a simple bug which can be fixed without an API-change:
	Check the status of the \code{trunk} version of the source file (use your most favorite SVN tool for that task and diff against your version).

	There is more than on possibility now.
	\begin{itemize}
	\item The bug was already fixed in \code{trunk}:
      Feel free to backport the bugfix to the release branch you're working on.
    \item The bug is not fixed in \code{trunk}.
	  Preferrably fix the bug in \code{trunk} first, then apply the same/similar fix to the release branch.
	\end{itemize}
	
  \item If fixing the error requires a change in the API or the way in which any given method operates (from the user's view), it \emph{must not} be commited to a release branch. Such changes always have to go into the current development version (\code{trunk}), and will probably make it into the next stable release. For the time being, you can use the development version of the library with the fix included. If you don't want that or it isn't possible due to whatever circumstances, you have to wait for the next official release to get the bugfix/feature.
  \end{itemize}
\end{itemize}


% TODO: this has to be moved to vrgroup internal documentation.
%The following sections give some guidelines in order to work with the SVN in a correct way.

%\subsection{Tags and Branches}
%% As a rule of thumb: create a tag each time you want to pinpoint a specific point in your development cycle, e.g., a successful demo or a fully tested state for your users.
%% In addition to that, try to remember what each tag stayed for.
%% A spreadsheet or note is something very useful here.
%% As for branches: try to avoid branches, unless you really want to care for more than one development trunk.
%% The ViSTA core lib sources are branched in the CVS in a HEAD and a BETA branch.
%% This is on purpose, as some features of the HEAD have to be evolved and are not for public access, whereas the BETA is a different flavor.
%% One could use a tag for the same effect and move the tag when a new state is reached. 
%% However, doing this in branches more clearly seperates directories when new files are added or old files are moved.
%% For tag or branch labels, try to be as expressive as possible in order to hint at users which revision to pick.
%% For example, when you finished a demo for a specific event, try to note the event in the tag label.
%% And please remember that tags can be removed, and you should do so, if you do not need them anymore.
