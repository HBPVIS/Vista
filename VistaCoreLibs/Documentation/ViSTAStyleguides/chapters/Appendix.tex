%% $Id$

%% Copyright (c)  1998-2013
%% by  RWTH-Aachen, Germany
%% Some rights reserved.

%% This work is licensed under the Creative Commons Attribution-Share
%% Alike 3.0 License. To view a copy of this license, visit
%% http://creativecommons.org/licenses/by-sa/3.0/ or send a letter to
%% Creative Commons, 171 Second Street, Suite 300, San Francisco,
%% California, 94105, USA.

\section{Appendix}
%\section{Document physical rules}
This section describes the physical layout for this specific document (the Styleguides themselves).
It comprises of topics dealing with files and resources that are of importance to this document.
You should respect the styles as depicted when applying changes to these Styleguides.

\subsection{Physical document layout}

The following subsections will deal with the topic of evolving the physical structure, e.g., file layout and the nomenclature within the passages.

\subsubsection{File layout}
Each \TeX\ file needs a proper preamble containing the license header.
A skeleton file is given in the file {\ttfamily template-for\-new-chapters.tex} file in this directory.
Note that it is mandatory to provide a proper SVN revision (Id) tag, this helps in identifying different versions of this document.

Each chapter lives in its own file in the subfolder \code{chapters}.
The name of this file should be related to the name of the chapter, but try to avoid whitespaces or irregular characters that might confuse a file system or \TeX.
For example, a file might be called {\ttfamily About.tex} while this chapter's heading is ''About this document''.

The language for this documentation is english.
Try to spellcheck as much as you can and try to avoid complicated sentences.

\subsubsection{Directory layout}
Graphics are to be placed in a subdirectory called {\ttfamily graphics}.
Any graphics for this document should be in moderate to high resolution and be placed in the directory {\ttfamily graphics}.
The format of the graphics has to be {\ttfamily eps}.

When checking new graphics into the repository, please keep in mind that {\ttfamily eps} figures have to be checked in as \textbf{binary} formatted.
See section \ref{sec:tools} for a reference on the tools that are to be used for the graphics output generation.

It is desired that any original graphics that was used to create an {\ttfamily eps} resource is kept in the revision control as well, just in case anybody wants to change it.
Those files should be kept in the subfolder {\ttfamily resources}.
Here it is valid to store the original file, e.g., PhotoShop {\ttfamily psd} or Visio sketches.
When in doubt, add a {\ttfamily README} for each or several file(s), including the extension, which describes the proper version to use when the resource is to be edited.

\subsection{\TeX coding}
In order to make the \TeX\ sources more diffable, do only use a newline command after every dot.
Especially, do not use automatic word wrap features of your favourite \TeX\ editor.
\begin{figure}
\begin{verbatim}
% right way to place endlines <edl>
This is an example sentence.<edl>
Newlines are always after a dot that ends the sentence.<edl>

% wrong way to place endlines
And this is another example sentence, that shows<edl>
a violation of this rule.<edl>
\end{verbatim}
\caption{\label{fig:texlines}
Example of the \TeX\ source formating (including a wrong way to do it).
Endlines do always end a sentence, even in captions.
See the source of this figure for a reference.}
\end{figure}

Please mark unfinished or work in progress sections with a proper, visually perceptible marking.
A \TeX\ macro called {\ttfamily todo} is provided with this document.
\todo{This is an example of the {\ttfamily todo} macro.}

It is used with the following code.
\begin{verbatim}
\todo{This is an example of the {\ttfamily todo} macro.}
\end{verbatim}

For figures that are not yet finished, use the {\ttfamily todo.eps} which is stored in the {\ttfamily graphics} directory.


\subsection{Tools}\label{sec:tools}

\subsubsection{\TeX nicscenter}
For \TeX\ editing, we recommend that you have the \TeX nicscenter available.
Project files for this editing environment are checked in to the repository for this document.
See http://www.toolscenter.org for more information.
Note that any text-editor with which you are comfortable will also be sufficient.

\subsubsection{JabRef}
For BibTeX editing, the JabRef tool is very nice.
See http://jabref.sourceforge.net for more information.
